\documentclass[]{article}

\usepackage[left=2.00cm, right=2.00cm, top=2.00cm, bottom=2.00cm]{geometry}
\usepackage[spanish,es-noshorthands]{babel}
\usepackage[utf8]{inputenc} % para tildes y ñ
\usepackage{graphicx} % para las figuras
\usepackage{xcolor}
\usepackage{listings} % para el código fuente en c++

\lstdefinestyle{customc}{
  belowcaptionskip=1\baselineskip,
  breaklines=true,
  frame=single,
  xleftmargin=\parindent,
  language=C++,
  showstringspaces=false,
  basicstyle=\footnotesize\ttfamily,
  keywordstyle=\bfseries\color{green!40!black},
  commentstyle=\itshape\color{gray!40!gray},
  identifierstyle=\color{black},
  stringstyle=\color{orange},
}
\lstset{style=customc}


%opening
\title{Práctica 2. Programación dinámica}
\author{Nombre Apellido1 Apellido2 \\ % mantenga las dos barras al final de la línea y este comentario
correo@servidor.com \\ % mantenga las dos barras al final de la linea y este comentario
Teléfono: xxxxxxxx \\ % mantenga las dos barras al final de la línea y este comentario
NIF: xxxxxxxxm \\ % mantenga las dos barras al final de la línea y este comentario
}


\begin{document}

\maketitle

%\begin{abstract}
%\end{abstract}

% Ejemplo de ecuación a trozos
%
%$f(i,j)=\left\{ 
%  \begin{array}{lcr}
%      i + j & si & i < j \\ % caso 1
%      i + 7 & si & i = 1 \\ % caso 2
%      2 & si & i \geq j     % caso 3
%  \end{array}
%\right.$

\begin{enumerate}
\item Formalice a continuación y describa la función que asigna un determinado valor a cada uno de los tipos de defensas.

$$ f(dmg, salud, APS, rango)=\frac{dmg}{APS} + 0.6 * salud + 0.1 * rango $$

Notas: dmg es el daño de la defensa y APS los ataques por segundo de la misma.

Como se puede ver, la función valora, sobre todo, la relación daño/APS, es decir, el DPS (damage per second) de la defensa, y en menor medida, la salud y el rango.


\item Describa la estructura o estructuras necesarias para representar la tabla de subproblemas resueltos.

Escriba aquí su respuesta al ejercicio 2.

\item En base a los dos ejercicios anteriores, diseñe un algoritmo que determine el máximo beneficio posible a obtener dada una combinación de defensas y \emph{ases} disponibles. Muestre a continuación el código relevante.

\begin{lstlisting}
void DEF_LIB_EXPORTED placeDefenses(bool** freeCells, int nCellsWidth, int nCellsHeight, float mapWidth, float mapHeight, std::list<Object*> obstacles, std::list<Defense*> defenses) {

    float cellWidth = mapWidth / nCellsWidth;
    float cellHeight = mapHeight / nCellsHeight;

    int maxAttemps = 1000;
    List<Defense*>::iterator currentDefense = defenses.begin();
    while(currentDefense != defenses.end() && maxAttemps > 0) {
        if(currentDefense == defenses.begin()){ //Colocar centro de extraccion --- Algoritmo devorador
            List<tipoCelda> C = getList0(nCellsWidth, nCellsHeight, mapWidth, mapHeight, obstacles);   //Conjunto de candidatos
            bool solucionado = false;
            while(!solucionado && !C.empty()){
                tipoCelda p = C.back();    //Como C esta ordenado de menor a mayor valor, la funcion de seleccion saca el elemento en la ultima posicion
                C.pop_back();              //Y la sacamos de la lista de candidatos
                if(factible(p.row, p.col, freeCells, nCellsWidth, nCellsHeight, mapWidth, mapHeight, obstacles, currentDefense, defenses)){
                    (*currentDefense)->position = p.position;   //Lo ponemos en la celda
                    freeCells[p.row][p.col] = false;
                    solucionado = true; //Problema solucionado
                }
            }
        }
        else{   //Colocar el resto de defensas --- Algoritmo devorador
            //...
        }

        ++currentDefense;
        maxAttemps--;
    }
}
\end{lstlisting}


\item Diseñe un algoritmo que recupere la combinación óptima de defensas a partir del contenido de la tabla de subproblemas resueltos. Muestre a continuación el código relevante.

Escriba aquí su respuesta al ejercicio 4.

\end{enumerate}

Todo el material incluido en esta memoria y en los ficheros asociados es de mi autoría o ha sido facilitado por los profesores de la asignatura. Haciendo entrega de este documento confirmo que he leído la normativa de la asignatura, incluido el punto que respecta al uso de material no original.

\end{document}
