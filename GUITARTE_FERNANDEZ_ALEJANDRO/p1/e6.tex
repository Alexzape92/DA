\begin{lstlisting}
void DEF_LIB_EXPORTED placeDefenses(bool** freeCells, int nCellsWidth, int nCellsHeight, float mapWidth, float mapHeight
              , std::list<Object*> obstacles, std::list<Defense*> defenses) {

    float cellWidth = mapWidth / nCellsWidth;
    float cellHeight = mapHeight / nCellsHeight;

    int maxAttemps = 1000;
    List<Defense*>::iterator currentDefense = defenses.begin();
    while(currentDefense != defenses.end() && maxAttemps > 0) {
        if(currentDefense == defenses.begin()){ //Colocar centro de extraccion --- Algoritmo devorador
            //Ejercicio 3
        }
        else{   //Colocar el resto de defensas --- Algoritmo devorador
            List<tipoCelda> C = getListRest(nCellsWidth, nCellsHeight, mapWidth, mapHeight, *(defenses.begin()));   //Conjunto de candidatos
            bool solucionado = false;
            while(!solucionado && !C.empty()){
                tipoCelda p = C.front();    //Como C esto ordenado de menor a mayor distancia, la funcion de seleccion saca el elemnto en la primera posicion
                C.pop_front();              //Y la sacamos de la lista de candidatos
                if(factible(p.row, p.col, freeCells, nCellsWidth, nCellsHeight, mapWidth, mapHeight, obstacles, currentDefense, defenses)){
                    (*currentDefense)->position = p.position;   //Lo ponemos en la celda
                    freeCells[p.row][p.col] = false;
                    solucionado = true; //Problema solucionado
                }
            }
        }

        ++currentDefense;
        maxAttemps--;
    }
}
\end{lstlisting}
