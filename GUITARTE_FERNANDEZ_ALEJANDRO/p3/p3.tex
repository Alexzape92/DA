\documentclass[]{article}

\usepackage[left=2.00cm, right=2.00cm, top=2.00cm, bottom=2.00cm]{geometry}
\usepackage[spanish,es-noshorthands]{babel}
\usepackage[utf8]{inputenc} % para tildes y ñ
\usepackage{graphicx} % para las figuras
\usepackage{xcolor}
\usepackage{listings} % para el código fuente en c++

\lstdefinestyle{customc}{
  belowcaptionskip=1\baselineskip,
  breaklines=true,
  frame=single,
  xleftmargin=\parindent,
  language=C++,
  showstringspaces=false,
  basicstyle=\footnotesize\ttfamily,
  keywordstyle=\bfseries\color{green!40!black},
  commentstyle=\itshape\color{gray!40!gray},
  identifierstyle=\color{black},
  stringstyle=\color{orange},
}
\lstset{style=customc}


%opening
\title{Práctica 3. Divide y vencerás}
\author{Nombre Apellido1 Apellido2 \\ % mantenga las dos barras al final de la línea y este comentario
correo@servidor.com \\ % mantenga las dos barras al final de la linea y este comentario
Teléfono: xxxxxxxx \\ % mantenga las dos barras al final de la línea y este comentario
NIF: xxxxxxxxm \\ % mantenga las dos barras al final de la línea y este comentario
}


\begin{document}

\maketitle

%\begin{abstract}
%\end{abstract}

% Ejemplo de ecuación a trozos
%
%$f(i,j)=\left\{ 
%  \begin{array}{lcr}
%      i + j & si & i < j \\ % caso 1
%      i + 7 & si & i = 1 \\ % caso 2
%      2 & si & i \geq j     % caso 3
%  \end{array}
%\right.$

\begin{enumerate}
\item Describa las estructuras de datos utilizados en cada caso para la representación del terreno de batalla. 

$$ f(dmg, salud, APS, rango)=\frac{dmg}{APS} + 0.6 * salud + 0.1 * rango $$

Notas: dmg es el daño de la defensa y APS los ataques por segundo de la misma.

Como se puede ver, la función valora, sobre todo, la relación daño/APS, es decir, el DPS (damage per second) de la defensa, y en menor medida, la salud y el rango.


\item Implemente su propia versión del algoritmo de ordenación por fusión. Muestre a continuación el código fuente relevante. 

Escriba aquí su respuesta al ejercicio 2.


\item Implemente su propia versión del algoritmo de ordenación rápida. Muestre a continuación el código fuente relevante. 

\begin{lstlisting}
void DEF_LIB_EXPORTED placeDefenses(bool** freeCells, int nCellsWidth, int nCellsHeight, float mapWidth, float mapHeight, std::list<Object*> obstacles, std::list<Defense*> defenses) {

    float cellWidth = mapWidth / nCellsWidth;
    float cellHeight = mapHeight / nCellsHeight;

    int maxAttemps = 1000;
    List<Defense*>::iterator currentDefense = defenses.begin();
    while(currentDefense != defenses.end() && maxAttemps > 0) {
        if(currentDefense == defenses.begin()){ //Colocar centro de extraccion --- Algoritmo devorador
            List<tipoCelda> C = getList0(nCellsWidth, nCellsHeight, mapWidth, mapHeight, obstacles);   //Conjunto de candidatos
            bool solucionado = false;
            while(!solucionado && !C.empty()){
                tipoCelda p = C.back();    //Como C esta ordenado de menor a mayor valor, la funcion de seleccion saca el elemento en la ultima posicion
                C.pop_back();              //Y la sacamos de la lista de candidatos
                if(factible(p.row, p.col, freeCells, nCellsWidth, nCellsHeight, mapWidth, mapHeight, obstacles, currentDefense, defenses)){
                    (*currentDefense)->position = p.position;   //Lo ponemos en la celda
                    freeCells[p.row][p.col] = false;
                    solucionado = true; //Problema solucionado
                }
            }
        }
        else{   //Colocar el resto de defensas --- Algoritmo devorador
            //...
        }

        ++currentDefense;
        maxAttemps--;
    }
}
\end{lstlisting}


\item Realice pruebas de caja negra para asegurar el correcto funcionamiento de los algoritmos de ordenación implementados en los ejercicios anteriores. Detalle a continuación el código relevante.

Escriba aquí su respuesta al ejercicio 4.

\item Analice de forma teórica la complejidad de las diferentes versiones del algoritmo de colocación de defensas en función de la estructura de representación del terreno de batalla elegida. Comente a continuación los resultados. Suponga un terreno de batalla cuadrado en todos los casos. 

Para el caso primero, en el que no ordenamos las defensas, obtendremos que el $t(n_{defs}, n_{cells}) \in \mathcal{O}(n_{defs} \cdot n_{cells}^{2})$, ya que en el bucle más externo recorremos todas las defensas, para colocarlas, y dentro de éste calculamos la lista ($n_{cells}$ para recorrer todas las celdas y $n_{cells}$ de nuevo en el peor caso para insertarla en la lista de manera ordenada. Esto es porque los elemenos se insertan de manera ordenada crecientemente, de manera que la función de selección simplemente elegirá el último elemento.

En el caso de la ordenación por fusión, el resultado será $t(n_{defs}, n_{cells}) \in \mathcal{O}(n_{defs} \cdot n_{cells} \cdot \log{n_{cells}})$. Se realiza el algorito de ordenación por fusión (complejidad $n \cdot \log{n}$ en el peor caso, y esto $n_{defs}$ veces.

En el caso de la ordenación rápida, el resultado será $t(n_{defs}, n_{cells}) \in \mathcal{O}(n_{defs} \cdot n_{cells} \cdot \log{n_{cells}})$. Se realiza el algorito de ordenación rápida (complejidad $n \cdot \log{n}$ en el caso promedio, y esto $n_{defs}$ veces.


\item Incluya a continuación una gráfica con los resultados obtenidos. Utilice un esquema indirecto de medida (considere un error absoluto de valor 0.01 y un error relativo de valor 0.001). Es recomendable que diseñe y utilice su propio código para la medición de tiempos en lugar de usar la opción \emph{-time-placeDefenses3} del simulador. Considere en su análisis los planetas con códigos 1500, 2500, 3500,..., 10500, al menos. Puede incluir en su análisis otros planetas que considere oportunos para justificar los resultados. Muestre a continuación el código relevante utilizado para la toma de tiempos y la realización de la gráfica.

\begin{lstlisting}
// sustituya este codigo por su respuesta
void placeDefenses(...) {

    List<Defense*>::iterator currentDefense = defenses.begin();
    while(currentDefense != defenses.end() && maxAttemps > 0) {

        (*currentDefense)->position.x = ((int)(_RAND2(nCellsWidth))) * cellWidth + cellWidth * 0.5f;
        ...
        ++currentDefense;
    }
}
\end{lstlisting}

\end{enumerate}

Todo el material incluido en esta memoria y en los ficheros asociados es de mi autoría o ha sido facilitado por los profesores de la asignatura. Haciendo entrega de este documento confirmo que he leído la normativa de la asignatura, incluido el punto que respecta al uso de material no original.

\end{document}
